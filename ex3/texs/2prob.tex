\section{Probability Inequalities}
\subsection{Markov Inequality}

Let $I_{X > c}$ be the indicator random variable of the event $X > c$:
\[
    I_{X> c} = \begin{cases}
                  1 & \text{if $X > c$,} \\
                  0 & \text{otherwise.}
                  \end{cases}
\]
We then have
\[
    cI_{X> c} = \begin{cases}
                   c & \text{if $X > c$,} \\
                   0 & \text{otherwise,}
                   \end{cases}
\]
and hence $$cI_{X> c} \leq X,$$ since $X$ is non-negative. 
Now, by the monotonicity of the expectation operator we have 
$$E[cI_{X> c}] \leq E[X].$$ Taking linearity of $E$, into account,
we get $$cE[I_{X> c}] \leq E[X].$$

Note that we can compute $E[I_{X> c}]$:
\begin{eqnarray*}
    E[I_{X> c}] &=& \sum_{x}p(x)I_{X> c} \\
                   &=& \sum_{x \leq c}p(x)I_{X> c} + \sum_{x > c}p(x)I_{X> c} \\
                   &=& \sum_{x \leq c}p(x)0 + \sum_{x > c}p(x)1 \\
                   &=& \sum_{x > c}p(x) \\
                   &=& P(X> x).
\end{eqnarray*}
Thus, we arrive at $$ cP(X > x) = cE[I_{X > c}] \leq E[X], $$ and hence
$$ P(X > x) \leq \frac{E[X]}{c}.$$

For the second part, let $X$ be the random variable, describing the number
of ``heads'' out of $n$ coin tosses. Since we are throwing a fair coin, we
clearly have $$ E[X] = \frac{n}{2}.$$ Now we have $c=\frac{3}{4}n$ and by the
Markov inequality we obtain 
\[ 
  P\left(X> \frac{3}{4}n\right) \leq \frac{\frac{n}{2}}{\frac{3}{4}n} 
    = \frac{2}{3}
\]
and $$ P\left(X > \frac{3}{4}n\right)\leq \frac{2}{3}.$$


\subsection{Chebyshev Inequality}
Similarly, let $I_{(X - E[X])^2 > c^2}$ be the indicator random variable of the event $(X - E[X])^2 > c^2$:
\[
    I_{(X - E[X])^2 > c^2} = \begin{cases}
                  1 & \text{if $(X - E[X])^2 > c^2$,} \\
                  0 & \text{otherwise.}
                  \end{cases}
\]
We then have
\[
    c^2I_{(X - E[X])^2 > c^2} = \begin{cases}
                   c^2 & \text{if $(X - E[X])^2 > c^2$,} \\
                   0 & \text{otherwise,}
                   \end{cases}
\]
and hence $$c^2I_{(X - E[X])^2 > c^2} \leq (X - E[X])^2.$$ 
Now, by the monotonicity of the expectation operator we have 
$$E[c^2I_{(X - E[X])^2 > c^2}] \leq E[(X-E[X])^2] = Var[E].$$ Taking linearity of $E$, into account,
we get $$c^2E[I_{(X - E[X])^2 > c^2}] \leq Var[E].$$

Note that we can compute $E[I_{(X - E[X])^2 > c^2}]$:
\begin{eqnarray*}
    E[I_{(X - E[X])^2 > c^2}] &=& \sum_{x}p(x)I_{(X - E[X])^2 > c^2}  \\
                              &=& P\left( (X-E[X])^2 > c^2 \right)    \\
                              &=& P\left( |X-E[X]| > c \right)
\end{eqnarray*}
Thus, we arrive at $$ P(|X-E[X]| > c) \leq \frac{Var[X]}{c^2}.$$

For the second part, let again $X$ be the random variable, describing the 
number of ``heads'' out of $n$ coin tosses. Now, due to the symmetry of the 
problem and the fact that $E[X]=\frac{n}{2}$, we make the following 
observation:
\[ P\left( X > \frac{3}{4}n \right) = P\left( X < \frac{1}{4}n \right).  \]

Thus, 
\begin{eqnarray}
P\left( X > \frac{3}{4}n \right) 
& = & \frac{2P\left(X > \frac{3}{4}n\right)}{2} \\
& = & \frac{P\left(X > \frac{3}{4}n\right) + P\left(X < \frac{1}{4}n\right)}{2} \\
& = & \frac{P\left(\{X > \frac{3}{4}n\} \cup \{X < \frac{1}{4}n\}\right)}{2} \\
& = & \frac{P\left( | X - E[X]| > \frac{1}{4}n \right)}{2}.
\end{eqnarray}

Now, after applying Chebyshev's inequality, we obtain:
\[
  P\left( X > \frac{3}{4}n \right) 
    \leq \frac{Var[X]}{2 \left(\frac{1}{4}n\right)^2}
    = \frac{8Var[X]}{n^2}.
\]

Now we just need the variance:
\begin{eqnarray*}
  Var[X] & = & E[(X-E[X])^2] \\
         & = & \sum_xp(x)\left(x-\frac{n}{2}\right)^2 \\
         & = & \sum_{k=0}^n\frac{\binom{n}{k}}{2^n}\left(k-\frac{n}{2}\right)^2.
\end{eqnarray*}


\subsection{Jensen's Inequality}
The statement is obviously true for $n=1$: 
$$ f(1\cdot x_1) \leq 1\cdot f(x_1)$$ and for $n=2$ since $f$ is convex:
$$ f(\lambda x_1 + (1-\lambda)x_2) \leq \lambda f(x_1) + (1-\lambda)f(x_2),$$
which gives us the base step of the induction.

Now suppose it is true for $n=k$ and consider $n=k+1$:
\begin{eqnarray}
  f\left(\sum_{i=1}^{k+1}\lambda_ix_i\right) 
    &=& f\left(\sum_{i=1}^{k}\lambda_ix_i+\lambda_{k+1}x_{k+1}\right) \\
    &=& f\left((1-\lambda_{k+1})\left(\frac{\sum_{i=1}^{k}\lambda_ix_i}{1-\lambda_{k+1}}\right)+\lambda_{k+1}x_{k+1}\right) \\
    &\leq& (1-\lambda_{k+1})f\left(\sum_{i=1}^k\frac{\lambda_ix_i}{1-\lambda_{k+1}}\right) + \lambda_{k+1}f(x_{k+1}) \\
    &\leq& (1-\lambda_{k+1})f\left(\sum_{i=1}^k\frac{\lambda_i}{1-\lambda_{k+1}}x_i\right) + \lambda_{k+1}f(x_{k+1}) \\
    &\leq& (1-\lambda_{k+1})\left(\sum_{i=1}^k\frac{\lambda_i}{1-\lambda_{k+1}}f(x_i)\right) + \lambda_{k+1}f(x_{k+1}) \\
    &=& \sum_{i=1}^k\lambda_if(x_i) + \lambda_{k+1}f(x_{k+1}) \\
    &=& \sum_{i=1}^{k+1}\lambda_if(x_i).
\end{eqnarray}

In going from (6) to (7) we used the definition of convexity of $f$. 
In going from (7) to (8) we use the fact that 
$$ \sum_{i=1}^k\lambda_i = 1 - \lambda_{k+1}$$ and apply the inductive 
hypothesis for $n=k$.
